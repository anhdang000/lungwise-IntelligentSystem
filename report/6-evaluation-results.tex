\section{Evaluation Results}

The performance of the XGBoost model was evaluated on the hold-out test set using standard evaluation metrics for classification problems.

\subsection{Key Performance Metrics}

The main performance metrics of the model on the test set are:
\begin{itemize}
    \item \textbf{Accuracy:} 0.91 (91.07\%). The ratio of total correct predictions to the total number of samples.
    \item \textbf{Precision:} For class 1 (Cancer): 0.90. The ratio of correctly predicted positive cases to all predicted positive cases (TP / (TP + FP)).
    \item \textbf{Recall / Sensitivity:} For class 1 (Cancer): 1.00. The ratio of correctly predicted positive cases to all actual positive cases (TP / (TP + FN)).
    \item \textbf{F1-Score:} For class 1 (Cancer): 0.95. The harmonic mean of Precision and Recall, providing a balanced measure between these two metrics.
    \item \textbf{Macro Avg:} Precision: 0.95, Recall: 0.79, F1-Score: 0.84. Average metrics calculated for each class separately.
    \item \textbf{Weighted Avg:} Precision: 0.92, Recall: 0.91, F1-Score: 0.90. Average metrics weighted by the number of samples in each class.
\end{itemize}

\subsection{Confusion Matrix}

The confusion matrix provides a more detailed view of model performance by breaking down correct and incorrect predictions for each class:

\begin{center}
\begin{tabular}{cc|c|c|}
  & \multicolumn{1}{c}{} & \multicolumn{2}{c}{Predicted Value} \\
  & \multicolumn{1}{c}{} & \multicolumn{1}{c}{No Cancer (0)}  & \multicolumn{1}{c}{Cancer (1)} \\
  \cline{3-4}
Actual Value & No Cancer (0) & \textbf{7} (TN) & 5 (FP) \\
  \cline{3-4}
  & Cancer (1) & 0 (FN) & \textbf{44} (TP) \\
  \cline{3-4}
\end{tabular}
\end{center}

Where:
\begin{itemize}
    \item \textbf{True Positives (TP):} 44 - Number of cancer cases correctly predicted.
    \item \textbf{False Negatives (FN):} 0 - Number of cancer cases incorrectly predicted as non-cancer.
    \item \textbf{False Positives (FP):} 5 - Number of non-cancer cases incorrectly predicted as cancer.
    \item \textbf{True Negatives (TN):} 7 - Number of non-cancer cases correctly predicted.
\end{itemize}

\subsection{Results Discussion}

The evaluation results show that the XGBoost model has excellent ability to distinguish between patients with high and low lung cancer risk. The overall accuracy is 91.07\%, showing that the model performs very well on the test dataset.

The most notable achievement is the perfect Recall score of 1.00 for the positive class (cancer), meaning the model successfully identified all actual cancer cases without any false negatives. This is particularly important in a medical context, where missing positive cases can have serious consequences.

The Precision for cancer prediction is 0.90, showing that among cases predicted as cancer, a small percentage (about 10\%) are incorrect predictions (False Positives). While the model has some room for improvement in reducing false positives, this trade-off is often acceptable in medical screening contexts where it's generally preferable to have some false alarms than to miss actual cases.

For the negative class (non-cancer), the precision is perfect (1.00), but the recall is lower (0.58), indicating that some non-cancer cases are incorrectly classified as cancer. This is reflected in the 5 false positives observed in the confusion matrix.

Overall, the model's performance is excellent, particularly for the critical task of identifying positive cancer cases. The high F1-score of 0.95 for cancer detection demonstrates a strong balance between precision and recall for the positive class. This model would provide valuable decision support for medical professionals in lung cancer screening contexts.
