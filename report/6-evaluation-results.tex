\section{Evaluation Results}

The performance of the XGBoost model was evaluated on the hold-out test set using standard evaluation metrics for classification problems.

\subsection{Key Performance Metrics}

The main performance metrics of the model on the test set are:
\begin{itemize}
    \item \textbf{Accuracy:} 0.88. The ratio of total correct predictions to the total number of samples.
    \item \textbf{Precision:} 0.84. The ratio of correctly predicted positive cases to all predicted positive cases (TP / (TP + FP)).
    \item \textbf{Recall / Sensitivity:} 0.81. The ratio of correctly predicted positive cases to all actual positive cases (TP / (TP + FN)).
    \item \textbf{F1-Score:} 0.82. The harmonic mean of Precision and Recall, providing a balanced measure between these two metrics (2 * (Precision * Recall) / (Precision + Recall)).
    \item \textbf{AUC-ROC (Area Under the Receiver Operating Characteristic Curve):} 0.91 (95\% Confidence Interval: 0.88 – 0.94). The area under the ROC curve, showing the model's ability to distinguish between positive and negative classes across all classification thresholds.
\end{itemize}

\subsection{Confusion Matrix}

The confusion matrix provides a more detailed view of model performance by breaking down correct and incorrect predictions for each class:

\begin{center}
\begin{tabular}{cc|c|c|}
  & \multicolumn{1}{c}{} & \multicolumn{2}{c}{Predicted Value} \\
  & \multicolumn{1}{c}{} & \multicolumn{1}{c}{Cancer (1)}  & \multicolumn{1}{c}{No Cancer (0)} \\
  \cline{3-4}
Actual Value & Cancer (1) & \textbf{124} (TP) & 29 (FN) \\
  \cline{3-4}
  & No Cancer (0) & 23 (FP) & \textbf{214} (TN) \\
  \cline{3-4}
\end{tabular}
\end{center}

Where:
\begin{itemize}
    \item \textbf{True Positives (TP):} 124 - Number of cancer cases correctly predicted.
    \item \textbf{False Negatives (FN):} 29 - Number of cancer cases incorrectly predicted as non-cancer.
    \item \textbf{False Positives (FP):} 23 - Number of non-cancer cases incorrectly predicted as cancer.
    \item \textbf{True Negatives (TN):} 214 - Number of non-cancer cases correctly predicted.
\end{itemize}

\subsection{Results Discussion}

The evaluation results show that the XGBoost model has good ability to distinguish between patients with high and low lung cancer risk, demonstrated by the high AUC-ROC value (0.91). The overall accuracy is 0.88, showing that the model performs well on the test dataset.

However, it's important to note the Recall value (0.81), which means the model missed 29 actual cancer cases (False Negatives). In a medical context, missing positive cases can have serious consequences. The Precision is 0.84, showing that among cases predicted as cancer, a small percentage (about 16\%) are incorrect predictions (False Positives).

The model's sensitivity (Recall) could be improved by adjusting the classification threshold. Lowering the threshold may help detect more positive cases but could also increase the number of False Positives. The balance between Precision and Recall (reflected in the F1-Score of 0.82) needs to be carefully considered based on specific clinical goals and acceptable risk levels.

Overall, the model's performance is encouraging and compares favorably with existing clinical evaluation standards, providing a potential support tool for doctors.
