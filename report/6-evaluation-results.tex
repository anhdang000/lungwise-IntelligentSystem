\section{Kết quả đánh giá}

Hiệu suất của mô hình XGBoost đã được đánh giá trên tập dữ liệu kiểm tra (hold-out set) bằng cách sử dụng các chỉ số đánh giá tiêu chuẩn cho bài toán phân loại.

\subsection{Các chỉ số đánh giá chính}

Các chỉ số hiệu suất chính của mô hình trên tập kiểm tra là:
\begin{itemize}
    \item \textbf{Độ chính xác (Accuracy):} 0.88. Tỷ lệ tổng số dự đoán đúng trên tổng số mẫu.
    \item \textbf{Độ chuẩn xác (Precision):} 0.84. Tỷ lệ các trường hợp được dự đoán là dương tính mà thực sự là dương tính (TP / (TP + FP)).
    \item \textbf{Độ phủ (Recall / Sensitivity):} 0.81. Tỷ lệ các trường hợp dương tính thực sự mà mô hình dự đoán đúng (TP / (TP + FN)).
    \item \textbf{Điểm F1 (F1-Score):} 0.82. Trung bình điều hòa của Precision và Recall, cung cấp một thước đo cân bằng giữa hai chỉ số này (2 * (Precision * Recall) / (Precision + Recall)).
    \item \textbf{AUC-ROC (Area Under the Receiver Operating Characteristic Curve):} 0.91 (Khoảng tin cậy 95%: 0.88 – 0.94). Diện tích dưới đường cong ROC, thể hiện khả năng phân biệt giữa các lớp dương tính và âm tính của mô hình trên tất cả các ngưỡng phân loại.
\end{itemize}

\subsection{Ma trận nhầm lẫn (Confusion Matrix)}

Ma trận nhầm lẫn cung cấp cái nhìn chi tiết hơn về hiệu suất của mô hình bằng cách phân tách các dự đoán đúng và sai cho từng lớp:

\begin{center}
\begin{tabular}{cc|c|c|}
  & \multicolumn{1}{c}{} & \multicolumn{2}{c}{Giá trị dự đoán} \\
  & \multicolumn{1}{c}{} & \multicolumn{1}{c}{Ung thư (1)}  & \multicolumn{1}{c}{Không ung thư (0)} \\
  \cline{3-4}
Giá trị thực tế & Ung thư (1) & \textbf{124} (TP) & 29 (FN) \\
  \cline{3-4}
  & Không ung thư (0) & 23 (FP) & \textbf{214} (TN) \\
  \cline{3-4}
\end{tabular}
\end{center}

Trong đó:
\begin{itemize}
    \item \textbf{True Positives (TP):} 124 - Số trường hợp ung thư được dự đoán đúng.
    \item \textbf{False Negatives (FN):} 29 - Số trường hợp ung thư bị dự đoán sai là không ung thư.
    \item \textbf{False Positives (FP):} 23 - Số trường hợp không ung thư bị dự đoán sai là ung thư.
    \item \textbf{True Negatives (TN):} 214 - Số trường hợp không ung thư được dự đoán đúng.
\end{itemize}

\subsection{Thảo luận kết quả}

Kết quả đánh giá cho thấy mô hình XGBoost có khả năng phân biệt tốt giữa các bệnh nhân có nguy cơ ung thư phổi cao và thấp, thể hiện qua giá trị AUC-ROC cao (0.91). Độ chính xác tổng thể là 0.88, cho thấy mô hình hoạt động tốt trên tập dữ liệu kiểm tra.

Tuy nhiên, cần lưu ý đến giá trị Recall (0.81), nghĩa là mô hình đã bỏ sót 29 trường hợp ung thư thực sự (False Negatives). Trong bối cảnh y tế, việc bỏ sót các trường hợp dương tính có thể gây hậu quả nghiêm trọng. Độ chuẩn xác (Precision) là 0.84, cho thấy trong số các trường hợp được dự đoán là ung thư, có một tỷ lệ nhỏ (khoảng 16%) là dự đoán sai (False Positives).

Có thể cải thiện độ nhạy (Recall) của mô hình bằng cách điều chỉnh ngưỡng phân loại (classification threshold). Việc giảm ngưỡng có thể giúp phát hiện nhiều trường hợp dương tính hơn nhưng cũng có thể làm tăng số lượng False Positives. Sự cân bằng giữa Precision và Recall (thể hiện qua F1-Score là 0.82) cần được xem xét cẩn thận dựa trên mục tiêu lâm sàng cụ thể và mức độ chấp nhận rủi ro.

Nhìn chung, hiệu suất của mô hình là đáng khích lệ và có thể so sánh thuận lợi với các tiêu chuẩn đánh giá lâm sàng hiện có, cung cấp một công cụ hỗ trợ tiềm năng cho các bác sĩ.
