\section{Conclusion}

The Lung Cancer Early Diagnosis and Monitoring application has been successfully developed, providing a comprehensive solution to support medical professionals in early disease detection and effective patient management. By integrating an XGBoost prediction model based on clinical data with an intuitive user interface and streamlined workflow, this system has the potential to significantly improve diagnostic accuracy and monitoring efficiency.

The evaluation results show that the XGBoost model achieves high performance in distinguishing high-risk patients, although adjustments are needed to optimize the balance between sensitivity and specificity in real clinical settings. The flexible system architecture, including React/TypeScript frontend, FastAPI/Python backend, and PostgreSQL database, ensures maintainability, scalability, and integration of new technologies in the future.

Future development directions include:
\begin{itemize}
    \item Expanding the training dataset with more samples and greater diversity to improve the model's generalization ability.
    \item Integrating additional data sources, especially medical imaging data (e.g., CT scans), to build multimodal models with higher accuracy.
    \item Continuously refining and re-evaluating the AI model as new data or more advanced algorithms become available.
    \item Conducting clinical trials to assess the actual impact of the application on doctor workflows and patient outcomes.
\end{itemize}

In summary, this project has laid a solid foundation for an AI-based clinical decision support tool, promising to bring practical benefits in the fight against lung cancer.
