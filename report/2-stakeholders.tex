\section{Các bên liên quan và lợi ích}
\begin{itemize}
    \item \textbf{Bác sĩ và nhân viên y tế:}
    \begin{itemize}
        \item \textit{Hỗ trợ quyết định lâm sàng:} Nhận diện nhanh chóng các bệnh nhân có nguy cơ ung thư phổi cao dựa trên phân tích AI, giúp tập trung nguồn lực và sự chú ý.
        \item \textit{Nâng cao độ chính xác chẩn đoán:} Tích hợp dự đoán AI với đánh giá lâm sàng để đưa ra chẩn đoán chính xác hơn, giảm thiểu sai sót.
        \item \textit{Tối ưu hóa quy trình làm việc:} Giảm thời gian cần thiết để xem xét hồ sơ bệnh án phức tạp nhờ giao diện tổng hợp thông tin trực quan.
        \item \textit{Cải thiện giao tiếp với bệnh nhân:} Có thêm công cụ để giải thích rõ ràng về nguy cơ và kế hoạch theo dõi cho bệnh nhân.
    \end{itemize}
    
    \item \textbf{Bệnh nhân:}
    \begin{itemize}
        \item \textit{Phát hiện sớm bệnh:} Tăng cơ hội phát hiện ung thư phổi ở giai đoạn đầu, cải thiện đáng kể tiên lượng và khả năng điều trị thành công.
        \item \textit{Kế hoạch theo dõi cá nhân hóa:} Nhận được lịch trình và phương pháp theo dõi phù hợp với mức độ nguy cơ và tình trạng sức khỏe cụ thể.
        \item \textit{Can thiệp kịp thời:} Giảm nguy cơ bệnh tiến triển nặng nhờ việc theo dõi sát sao và can thiệp y tế đúng lúc.
        \item \textit{Tăng cường sự an tâm:} Hiểu rõ hơn về tình trạng sức khỏe của mình và cảm thấy yên tâm hơn với quy trình chăm sóc chủ động.
    \end{itemize}

    \item \textbf{Quản trị viên bệnh viện và phòng khám:}
    \begin{itemize}
        \item \textit{Giám sát hiệu quả hoạt động:} Truy cập các báo cáo và bảng điều khiển tổng hợp để theo dõi hiệu quả của chương trình sàng lọc và chẩn đoán.
        \item \textit{Phân bổ nguồn lực tối ưu:} Đưa ra quyết định dựa trên dữ liệu về việc phân bổ nhân lực, thiết bị và ngân sách cho hoạt động chăm sóc bệnh nhân ung thư phổi.
        \item \textit{Đảm bảo và nâng cao chất lượng dịch vụ:} Sử dụng dữ liệu hiệu suất để chứng minh sự tuân thủ các tiêu chuẩn chất lượng y tế và xác định các lĩnh vực cần cải thiện.
        \item \textit{Hỗ trợ lập kế hoạch chiến lược:} Cung cấp thông tin chi tiết về xu hướng bệnh tật và hiệu quả can thiệp, hỗ trợ xây dựng chiến lược y tế dài hạn.
    \end{itemize}

    \item \textbf{Nhà khoa học dữ liệu và nhà phát triển:}
    \begin{itemize}
        \item \textit{Tinh chỉnh và cải tiến mô hình AI:} Kiến trúc mô-đun và quy trình dữ liệu rõ ràng cho phép dễ dàng thử nghiệm, đánh giá và cập nhật các mô hình dự đoán.
        \item \textit{Phát triển và mở rộng tính năng:} Nhanh chóng tích hợp các nguồn dữ liệu mới (ví dụ: hình ảnh y tế) hoặc thêm các chức năng hỗ trợ quyết định khác.
        \item \textit{Triển khai liền mạch các dịch vụ AI mới:} Dễ dàng tích hợp các thuật toán hoặc công nghệ AI tiên tiến khác vào hệ thống hiện có.
        \item \textit{Bảo trì và nâng cấp hệ thống hiệu quả:} Thiết kế hệ thống rõ ràng giúp đơn giản hóa việc sửa lỗi, bảo trì và nâng cấp phần mềm.
    \end{itemize}

    \item \textbf{Cơ quan quản lý và công ty bảo hiểm:}
    \begin{itemize}
        \item \textit{Đảm bảo tuân thủ quy định:} Cung cấp dấu vết kiểm toán minh bạch và tài liệu hóa quy trình để chứng minh sự tuân thủ các quy định về quyền riêng tư dữ liệu (ví dụ: HIPAA, GDPR) và an toàn y tế.
        \item \textit{Xác thực độ tin cậy của mô hình:} Truy cập các chỉ số hiệu suất tổng hợp để đánh giá tính công bằng, độ chính xác và độ tin cậy của các thuật toán AI được sử dụng.
        \item \textit{Đánh giá hiệu quả chi phí:} Phân tích dữ liệu để xác định lợi tức đầu tư và hiệu quả chi phí của việc triển khai hệ thống chẩn đoán sớm dựa trên AI.
        \item \textit{Quản lý rủi ro và chính sách:} Thông tin từ hệ thống có thể hỗ trợ việc xây dựng các chính sách y tế và bảo hiểm dựa trên bằng chứng.
    \end{itemize}
\end{itemize}
